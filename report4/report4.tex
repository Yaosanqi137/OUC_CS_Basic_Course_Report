
\documentclass[UTF8]{gyh}

\usepackage{amsmath}
\usepackage{cases}
\usepackage{cite}
\usepackage[margin=1in]{geometry}
\geometry{a4paper}
\usepackage{fancyhdr}
\usepackage{listings} % 添加 listings 宏包以使用 lstlisting 环境
\pagestyle{fancy}
\usepackage{graphicx}
\usepackage{indentfirst}
\usepackage{float}
\usepackage[hidelinks]{hyperref}
\usepackage{xcolor}   % 推荐添加,用于代码颜色

\lstset{
extendedchars=\true, % 开启扩展字符集,支持UTF-8
literate=
{<}{\textless}1
{>}{\textgreater}1,
frame=single               % 添加边框
}

\title{系统开发工具基础实验报告四}
\author{古宇恒}
\date{\today}
\pagenumbering{arabic}

\begin{document}

\fancyhead[C]{性能分析调试及Pytorch入门}
\fancyfoot[C]{\thepage}

\maketitle
\tableofcontents
\newpage

\section{摘要}
本实验报告旨在深入探讨现代软件开发中的关键工具与技术,主要涵盖三大核心领域:性能分析与调试、元编程与自动化构建,以及流行的深度学习框架 PyTorch 的基础知识。通过本报告的梳理与总结,期望能为后续的系统开发实践奠定坚实的基础。

\section{第一部分:性能分析与调试}
\subsection{调试的重要性}
在编程中,代码很少能一次性完美运行。调试是定位并修复代码中错误(Bug)的关键过程。一个高效的调试流程能极大缩短开发周期,提高代码质量。所有程序员,无论经验水平如何,都会花费大量时间进行调试。因此,掌握系统化的调试策略和工具至关重要。

\subsection{打印调试法}
最简单直接的调试方法是在代码的关键位置插入 print 语句,输出变量值或程序执行状态,从而追踪程序的运行逻辑。

\begin{lstlisting}
def my_function(data):
result = 0
for i, value in enumerate(data):
print(f"Iteration {i}: current value = {value}, result = {result}")
result += value
return result

my_function()
\end{lstlisting}

\subsection{日志系统}
相比于临时添加的 print 语句,日志系统更为强大和规范。

\begin{lstlisting}
import logging

logging.basicConfig(level=logging.INFO, format='%(asctime)s - %(levelname)s - %(message)s')

logging.debug("这是一条调试信息,通常用于开发阶段。")
logging.info("程序正常运行。")
logging.warning("出现了一个小问题,但不影响核心功能。")
logging.error("发生了一个错误,程序可能无法继续。")
\end{lstlisting}

\subsection{为日志着色}
为了提高日志的可读性,可以使用 ANSI escape codes 为终端输出添加颜色。

\begin{lstlisting}

# 打印红色的 "Error:"
echo -e "\e[31mError:\e[0m Something went wrong."

# 打印绿色的 "Success:"
echo -e "\e[32mSuccess:\e[0m Task completed."

# 打印黄色的 "Warning:"
echo -e "\e[33mWarning:\e[0m Please check the configuration."
\end{lstlisting}

\subsection{系统日志}
现代操作系统通常有统一的系统日志服务。在 Linux 中,systemd 将日志记录在 /var/log/journal,可通过 journalctl 查看。

\begin{lstlisting}

# 查看所有systemd日志
journalctl

# 查看特定服务的日志(例如 sshd)
journalctl -xeu sshd

# 直接使用 systemctl 也可以
systemctl status sshd
\end{lstlisting}

\photo{1}{img/img.png}{systemctl示例}

\photo{1}{img/img_1.png}{journalctl示例}

\subsection{交互式调试器}
当打印调试不足以解决问题时,就需要使用调试器,如 Python 的 pdb 或 C/C++ 的 gdb。

\begin{lstlisting}

# 在命令行中直接对脚本 my_script.py 启动pdb调试
python -m pdb my_script.py
\end{lstlisting}

\subsection{pdb 常用命令}
\begin{itemize}
\item l(ist): 显示当前行附近的代码。
\item n(ext): 执行到下一行(函数调用被视为一步)。
\item s(tep): 单步执行,如果当前是函数调用,则进入函数内部。
\item b(reak) [lineno]: 在指定行号设置断点。
\item c(ontinue): 继续执行,直到遇到下一个断点。
\item p(rint) [expression]: 打印表达式的值。
\item q(uit): 退出调试器。
\end{itemize}

\subsection{静态分析}
静态分析工具可以在不运行代码的情况下检查代码中的潜在错误,如变量覆盖、未定义变量、类型不匹配等。

\begin{lstlisting}

# 假设有一个名为 main.py 的文件
# 使用 mypy 进行类型检查
mypy main.py

# 使用 pyflakes 检查潜在错误
pyflakes main.py
\end{lstlisting}

\subsection{代码风格检查}
Linter 是一种静态分析工具,用于检查代码是否符合特定的编码规范和风格指南。Python 中常用的工具有 pylint 和 pep8。

\begin{lstlisting}
pylint your_module.py
\end{lstlisting}

\subsection{CPU 分析器}
\begin{itemize}
\item 追踪分析器 (Tracing Profiler): 记录每一次函数调用。虽然精确,但会带来较大的性能开销。
\item 采样分析器 (Sampling Profiler): 周期性地检查程序堆栈,性能开销小,适用于生产环境。
\end{itemize}
Python 内置的 cProfile 是一个确定性分析器,它会记录所有函数调用。

\begin{lstlisting}

# 对脚本 a_script.py 进行性能分析,并将结果按累计时间排序
python -m cProfile -s cumulative a_script.py
\end{lstlisting}

\subsection{内存分析}
内存分析器用于检测内存泄漏或不合理的内存使用。对于 Python 这类有垃圾回收机制的语言,它可以帮助我们发现那些因为存在引用而无法被回收的对象。

\begin{lstlisting}

# 在需要分析的函数上添加 @profile 装饰器
from memory_profiler import profile

@profile
def create_large_list():
return [i for i in range(1000000)]

if name == 'main':
create_large_list()

# 然后在命令行中运行: python -m memory_profiler your_script_name.py
\end{lstlisting}

\subsection{可视化分析工具}
\begin{itemize}
\item 火焰图 (Flame Graph): 一种可交互的可视化工具,用于展示采样分析器的数据,能直观地显示出函数调用的耗时比例。火焰图的宽度代表函数在总耗时中的占比,堆栈的深度则代表函数调用的层级。
\end{itemize}
\begin{lstlisting}

# 安装 py-spy: pip install py-spy
# 对一个正在运行的Python进程(PID)进行采样并生成火焰图
sudo py-spy record -p <PID> --output profile.svg
\end{lstlisting}

\section{第二部分:元编程与自动化}

\subsection{构建系统}
构建系统用于自动化执行从源码到最终产品的多步骤过程。make 是一个经典的构建系统,它通过 Makefile 文件来定义目标、依赖和构建规则。当依赖项比目标更新时,make 会自动执行相应的命令来重新生成目标。

\begin{lstlisting}

# Makefile
paper.pdf: paper.tex plot.png
pdflatex paper.tex

plot.png: data.csv plot.py
python plot.py -i data.csv -o plot.png

clean:
rm -f *.aux *.log paper.pdf plot.png
\end{lstlisting}

\subsection{依赖管理}
现代软件开发严重依赖于第三方库。依赖管理工具(如 Python 的 pip、Node.js 的 npm)帮助我们自动下载、安装和管理这些依赖项。一个常见的做法是使用 requirements.txt 文件来声明项目的所有 Python 依赖。

\begin{lstlisting}

# requirements.txt
numpy==1.21.0
pandas>=1.3.0
matplotlib
\end{lstlisting}

\begin{lstlisting}
pip install -r requirements.txt
\end{lstlisting}

\subsection{语义化版本控制}
为了解决依赖更新可能带来的兼容性问题,语义化版本控制(格式为 主版本号.次版本号.修订号)被广泛采用。

\subsection{持续集成 (CI)}
CI 是一种自动化流程,当代码仓库发生变更时,会自动执行构建、测试、部署等一系列操作。这确保了新的代码提交不会破坏现有功能,并能及早发现问题。GitHub Actions 是一个流行的 CI/CD 工具。

\begin{lstlisting}

# .github/workflows/ci.yml
name: Python CI

on: [push, pull_request]

jobs:
build:
runs-on: ubuntu-latest
steps:
- uses: actions/checkout@v2
- name: Set up Python
uses: actions/setup-python@v2
with:
python-version: '3.8'
- name: Install dependencies
run: |
python -m pip install --upgrade pip
pip install -r requirements.txt
- name: Test with pytest
run: |
pytest
\end{lstlisting}

\subsection{软件测试的类型}
\begin{itemize}
\item 单元测试 (Unit Test): 对代码中最小的可测试单元(如函数、类)进行测试。
\item 集成测试 (Integration Test): 测试多个组件或模块协同工作时是否符合预期。
\item 回归测试 (Regression Test): 确保新的代码修改没有引入新的错误或导致旧的功能失效。
\end{itemize}
\begin{lstlisting}
import unittest

def add(a, b):
return a + b

class TestAdd(unittest.TestCase):
def test_add_integers(self):
self.assertEqual(add(1, 2), 3)

 code Codedownloadcontent_copy
expand_less

    def test_add_floats(self):
    self.assertAlmostEqual(add(0.1, 0.2), 0.3)




if name == 'main':
unittest.main()
\end{lstlisting}

\section{第三部分:PyTorch 入门}
\subsection{PyTorch 简介}
PyTorch 是由 Facebook 开发的开源机器学习库,以其灵活性和动态计算图(Dynamic Computational Graph)而闻名,特别适合于研究和快速原型开发。它与 Python 紧密集成,语法直观,深受学术界和工业界的喜爱。

\subsection{张量}
张量是 PyTorch 中最基本的数据结构,类似于 NumPy 的 ndarray,但增加了在 GPU 上进行计算的能力。

\begin{lstlisting}
import torch

# 创建一个未初始化的 5x3 矩阵
x = torch.empty(5, 3)
print(x)

# 创建一个随机初始化的矩阵
x = torch.rand(5, 3)
print(x)

# 创建一个全零矩阵并指定数据类型
x = torch.zeros(5, 3, dtype=torch.long)
print(x)

# 加法操作
y = torch.rand(5, 3)
print(x + y)
\end{lstlisting}

\subsection{自动微分}
PyTorch 的 autograd 模块为张量上的所有操作提供自动微分功能。通过设置 requires_grad=True,PyTorch 会自动追踪其上的所有操作,并能自动计算梯度,这是构建神经网络的核心。当调用 .backward() 时,计算图会自动计算所有需要梯度的张量相对于该变量的梯度。

\begin{lstlisting}
import torch

# 创建一个张量并设置 requires_grad=True 来追踪计算}
x = torch.ones(2, 2, requires_grad=True)
print(x)

# 对张量进行操作
y = x + 2
z = y * y * 3
out = z.mean()

# 计算梯度
out.backward()

# 打印梯度 d(out)/dx}
print(x.grad)
\end{lstlisting}

\subsection{神经网络模块}
torch.nn 模块提供了构建神经网络所需的所有基本组件,如各种网络层(线性层、卷积层等)、激活函数(ReLU、Sigmoid 等)和损失函数。通过继承 nn.Module,可以方便地定义自己的网络结构。

\begin{lstlisting}
import torch
import torch.nn as nn
import torch.nn.functional as F

class Net(nn.Module):
    def init(self):
        super(Net, self).init()
        # 1 input image channel, 6 output channels, 5x5 square convolution
        self.conv1 = nn.Conv2d(1, 6, 5)
        self.pool = nn.MaxPool2d(2, 2)
        self.fc1 = nn.Linear(6 * 12 * 12, 120)
        self.fc2 = nn.Linear(120, 84)
        self.fc3 = nn.Linear(84, 10)

    def forward(self, x):
        x = self.pool(F.relu(self.conv1(x)))
        x = x.view(-1, 6 * 12 * 12)
        x = F.relu(self.fc1(x))
        x = F.relu(self.fc2(x))
        x = self.fc3(x)
        return x

net = Net()
print(net)
\end{lstlisting}

\subsection{优化器}
torch.optim 模块实现了各种优化算法,如 SGD、Adam、RMSprop 等,用于在训练过程中更新模型的权重。优化器接收模型参数和学习率等超参数,并在 optimizer.step() 调用时更新参数。

\begin{lstlisting}
import torch.optim as optim
import torch.nn as nn

# 假设 net 是我们定义的 nn.Module 实例
net = Net()

# 定义损失函数和优化器
criterion = nn.CrossEntropyLoss()
optimizer = optim.SGD(net.parameters(), lr=0.01, momentum=0.9)

# 在训练循环中:
optimizer.zero_grad() # 清除旧的梯度
loss.backward()       # 反向传播计算梯度
optimizer.step()      # 更新权重
\end{lstlisting}

\subsection{数据加载与预处理}
PyTorch 提供了 Dataset 和 DataLoader 类,用于方便地加载、预处理和批量化数据,极大地简化了数据处理流程。Dataset 负责封装数据源和样本的获取,而 DataLoader 则负责将 Dataset 包装成一个可迭代对象,并支持多线程加载、数据打乱和批处理。

\begin{lstlisting}
import torch
from torch.utils.data import TensorDataset, DataLoader

# 加载特征和标签张量
features = torch.randn(100, 10) # 100个样本, 10个特征
labels = torch.randint(0, 2, (100,)) # 100个标签

# 创建 Dataset
dataset = TensorDataset(features, labels)

# 创建 DataLoader
batch_size=10 每次迭代加载10个样本
shuffle=True 表示在每个epoch开始时打乱数据顺序
dataloader = DataLoader(dataset, batch_size=10, shuffle=True)

# 遍历数据
for batch_features, batch_labels in dataloader:
    print("Batch Features Shape:", batch_features.shape)
    print("Batch Labels Shape:", batch_labels.shape)
\end{lstlisting}


\end{document}